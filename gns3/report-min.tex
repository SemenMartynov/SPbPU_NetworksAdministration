\documentclass[a4paper, 12pt]{article}		% general format

%%%% Charset
\usepackage{cmap}							% make PDF files searchable and copyable
\usepackage[utf8x]{inputenc}				% accept different input encodings
\usepackage[T2A]{fontenc}					% russian font
\usepackage[russian]{babel}					% multilingual support (T2A)

%%%% Graphics
\usepackage[dvipsnames]{xcolor}			% driver-independent color extensions
\usepackage{graphicx}						% enhanced support for graphics
\usepackage{wrapfig}						% produces figures which text can flow around

%%%% Math
\usepackage{amsmath}						% American Mathematical Society (AMS) math facilities
\usepackage{amsfonts}						% fonts from the AMS
\usepackage{amssymb}						% additional math symbols

%%%% Typograpy (don't forget about cm-super)
\usepackage{microtype}						% subliminal refinements towards typographical perfection
\linespread{1.3}							% line spacing
\usepackage[left=2.5cm, right=1.5cm, top=2.5cm, bottom=2.5cm]{geometry}
\setlength{\parindent}{0pt}					% we don't want any paragraph indentation
\usepackage{parskip}						% some distance between paragraphs

%%%% Tables
\usepackage{tabularx}						% tables with variable width columns
\usepackage{multirow}						% for tabularx
\usepackage{hhline}							% for tabularx

%%%% Graph
\usepackage{tikz}							% package for creating graphics programmatically
\usetikzlibrary{arrows}						% edges for tikz

%%%% Other
\usepackage{url}							% verbatim with URL-sensitive line breaks
\usepackage{fancyvrb}						% sophisticated verbatim text (with box)
%------------------------------------------------------------------------------

\begin{document}

%------------------------------------------------
\begin{titlepage}
\thispagestyle{empty}

\begin{center}
Санкт-Петербургский политехнический университет Петра Великого\\
Институт Информационных Технологий и Управления \\*
Кафедра компьютерных систем и программных технологий \\*
\hrulefill
\end{center}

\vspace{15em}

\begin{center}
\Large Отчёт по индивидуальному заданию\\по предмету «Администрирование компьютерных сетей» \\
\end{center}

\vspace{1em}

% \linebreak
\begin{center}
\textsc{\textbf{Graphical Network Simulator-3}}
\end{center}

\vspace{20em}

\begin{flushleft}
Работу выполнил студент гр. 53501/3 \hrulefill Мартынов С. А. \\
\vspace{1.5em}
Работу принял преподаватель \hrulefill Малышев И. А. \\
\end{flushleft}

\vspace{\fill}

\begin{center}
Санкт-Петербург \\
2015
\end{center}

\end{titlepage}
%------------------------------------------------
\setcounter{page}{2}

\textbf{Введение}

Проект GNS3 является самым популярным решением для эмуляции сетевого оборудования. На данный момент это графический эмулятор сети, который позволяет смоделировать виртуальную сеть из маршрутизаторов и виртуальных машин.

\textbf{Установка}

Установить GNS3 можно с официального сайта - \url{http://www.gns3.com/}. Для deb-like систем (Debian, Ubuntu, Mint) GNS3 доступен прямо из репозитория.

\begin{Verbatim}[frame=single]
apt-get install gns3
\end{Verbatim}

Первый запуск программы сопровождается появлением окна c мастером установки, состоящим из трёх шагов:
\begin{enumerate}
  \item указание путей к образом ОС;
  \item проверка корректности путей из шага 1;
  \item добавление образов IOS.
\end{enumerate}

Перейдём к добавлению образов IOS. Их можно легко найти в интернете, на момент написания работы богатый выбор представлен на сайте \url{http://certs4u.info/ciscoios/c2650/c2691/}. Диалог добавления образа находится в меню Edit, пункт IOS images and hypervisors. Мы используем файл c2691-adventerprisek9\_sna-mz.124-13b.bin, а система автоматически определяет модель маршрутизатора 

Теперь можно подключиться к работающему роутеру. Для подключения, нужно набрать следующую команду, после чего появится доступ к управлению роутером.
\begin{Verbatim}[frame=single]
telnet 127.0.0.1 2101
\end{Verbatim}

\textbf{Виртуальные машины в эмулируемой сети}

GNS3 можно соединить с хостовой ОС Linux (на котором он запущен) и серверами в VirtualBox-е. Для реализации этой схемы подключения, произведём установку утилиты tunctl для создания и управления TUN/TAP виртуальными сетевыми интерфейсами:

\begin{Verbatim}[frame=single]
ubox$ sudo apt-get install -y uml-utilities
\end{Verbatim}

И утилиту brctl для создания и настройки сетевых мостов:
\begin{Verbatim}[frame=single]
ubox$ sudo apt-get install -y bridge-utils
\end{Verbatim}

Теперь создаем и конфигурируем виртуальные сетевые интерфейсы:
\begin{itemize}
\item tap0 -- для связи с Linux-ом, на котором и запущен GNS3.
\item tap1 -- для связи через мост с гостевыми машинами VirtualBox-а.
\end{itemize}
\begin{Verbatim}[frame=single]
ubox$ sudo tunctl -t tap0 -u username
ubox$ sudo tunctl -t tap1 -u username
ubox$ sudo ip addr add 192.168.1.1/24 dev tap0
ubox$ sudo ip link set up dev tap0
\end{Verbatim}

Обеспечим привязку интерфейсов к облаку. Связь с VirtualBox-ом осуществляется через мост br0, который состоит из виртуального Host-only интерфейса vboxnet0 и уже созданного tap1.
\begin{Verbatim}[frame=single]
ubox$ sudo brctl addbr br0
ubox$ sudo brctl addif br0 tap1
ubox$ sudo brctl addif br0 vboxnet0
ubox$ sudo ifconfig br0 192.168.3.4 netmask 255.255.255.0 up
\end{Verbatim}

Для связи всего этого хозяйства c хостовым Linux-ом на нем необходимо прописать маршрутизацию в используемые подсети:
\begin{Verbatim}[frame=single]
ubox$ sudo route add -net 10.1.1.0/24 gw 192.168.1.1
ubox$ sudo route add -net 10.2.2.0/24 gw 192.168.1.1
ubox$ sudo route add -net 192.168.3.0/24 gw 192.168.1.1
\end{Verbatim}

Теперь перейдём к настройке роутеров. На всех роутерах нужно прописать роутинг на подсети (или использовать протоколы динамической маршрутизации).

Для R1 --

\begin{Verbatim}[frame=single]
R1# conf t
R1(config)# router eigrp 1
R1(config-router)# passive-interface FastEthernet0/0
R1(config-router)# network 10.1.1.0 0.0.0.3
R1(config-router)# network 192.168.1.0
R1(config-router)# no auto-summary
R1(config-router)# exit
R1(config)# ip route 0.0.0.0 0.0.0.0 FastEthernet0/0
\end{Verbatim}

Для R2 --

\begin{Verbatim}[frame=single]
R2# conf t
R2(config)# router eigrp 1
R2(config-router)# network 10.1.1.0 0.0.0.3
R2(config-router)# network 10.2.2.0 0.0.0.3
R2(config-router)# no auto-summary
R2(config-router)# exit
R2(config)# ip route 0.0.0.0 0.0.0.0 Serial0/0
\end{Verbatim}

Для R3 --

\begin{Verbatim}[frame=single]
R3# conf t
R3(config)# router eigrp 1
R3(config-router)# passive-interface FastEthernet0/0
R3(config-router)# network 10.2.2.0 0.0.0.3
R3(config-router)# network 192.168.3.0
R3(config-router)# no auto-summary
R3(config-router)# exit
R3(config)# ip route 0.0.0.0 0.0.0.0 Serial0/0
\end{Verbatim}

На Debian-е (запущенном в VirtualBox) устанавливается сетевой адрес и шлюз по умолчанию:
\begin{Verbatim}[frame=single]
debianbox$ ifconfig eth0 192.168.3.3 netmask 255.255.255.0 up
debianbox$ route add default gw 192.168.3.1
\end{Verbatim}

Для того, чтобы работал Интернет через хостовый Linux нужно применить следующие правила (eth0 — внешний интерфейс).
\begin{Verbatim}[frame=single]
echo 1 > /proc/sys/net/ipv4/ip_forward
ubox$ sudo /sbin/iptables -t nat -A POSTROUTING -o eth0 -j MASQUERADE
ubox$ sudo /sbin/iptables -t nat -A POSTROUTING -o eth0 -j LOG
ubox$ sudo /sbin/iptables -A FORWARD -i eth0 -o tap0 -m state --state 
                                             RELATED,ESTABLISHED -j ACCEPT
ubox$ sudo /sbin/iptables -A FORWARD -i tap0 -o eth0 -j ACCEPT
\end{Verbatim}

На базе данного примера можно строить сетевые топологии еще больше и сложнее.

\textbf{Заключение}

GNS3 является прекрасным инструментом для проведения лабораторных работ и обучению всем сетевым премудростям Cisco. Данная программа подходит для самостоятельной подготовки к сдаче экзаменов по Cisco -- CCNA, CCNP, CCIP и CCIE.

\end{document}