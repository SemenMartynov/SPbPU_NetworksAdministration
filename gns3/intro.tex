\newpage
\section*{Введение}
\addcontentsline{toc}{section}{Введение}

Эмуляция на пользовательском уровне начала своё бурное развитие в начале 2000-х годов силами компании VMware. В начале был представлен VMware Workstation, чуть позже VMware GSX Server и VMware ESX Server. Но ситуация с эмуляторами сетевого оборудования на тот момент выглядела достаточно плачевно.
В августе 2005 года Кристоф Филот (Christophe Fillot) представил свой эмулятор Dynamips, который позволяет эмулировать аппаратную часть маршрутизаторов, непосредственно загружая и взаимодействуя с реальными образами Cisco IOS. 

Проект постепенно развивался, но в 2007 году был заброшен; последняя версия от первоначального автора была с номером 0.2.8-RC2. При этом проект был явно далёк от завершения и имел ряд достаточно неудобных решений: в текстовых конфигурационных файлах приходилось описывать всю топологию вручную.

Позже, в 2007 году, Джереми Гроссман (Jeremy Grossman) начал разработку GNS3 (Graphical Network Simulator-3), в качестве своего дипломного проекта во время учебы в университете.

Со временем GNS3 стал самым популярным решением для эмуляции сетевого оборудования. На данный момент это графический эмулятор сети, который позволяет смоделировать виртуальную сеть из маршрутизаторов и виртуальных машин. Широко используемый инструмент для обучения и тестов. Работает практически на всех платформах. Подходит для создания стендов на десктоп машинах. Проект пользуется поддержкой от Intel, at\&t, Cisco, Verizon, sprint, IBM, Alcatel-Lucent, Huawei, HP и т.д.

В случае отсутствия доступа к реальному оборудованию, GNS3 становится практически полноценной лабораторией. Кроме того, лабораторные работы выполняемые в GNS3, могут стать дополнением к занятиям в реальной лаборатории студентам готовящимся к сертификационным экзаменам CCNA/CCNP и CCIE

Единственным недостатком GNS3  является отсутствие возможности полноценной эмуляции коммутаторов второго уровня Cisco. Данный недостаток по заявлению разработчиков не может исправлен в следующих версиях, так как его причиной является кардинальное различие в аппаратной платформе маршрутизаторов и свитчей Cisco. В некоторых случаях данный недостаток получается обойти при помощи сетевого модуля NM-16ESW.

В состав GNS3 не входят образы IOS/IPS/PIX/ASA/JunOS, так как они являются частью коммерческих продуктов соответствующих компаний, и никакого прямого отношения к проекту GNS3 не имеют. Тем не менее, их поиск не составляет труда.

Одной из самых интересных особенностей GNS3 является возможность соединения проектируемой топологии с реальной физической сетью. Это дает возможность проверить на практике какой-либо проект, без использования реального оборудования. Использование WireShark позволяет провести мониторинг трафика внутри проектируемой топологии, и даёт дополнительную информацию для понимания изучаемых технологий.

Важной особенностью является и то, что GNS3 бесплатный продукт. Это открытое программное обеспечение, и любой желающий может скачать его с официального сайта проекта в разделе Download. Имеются версии для Linux, MS Windows XP и Windows 7, а также для Mac OS. На момент написания работы, актуальной версией является 1.3.3.

В зависимости от аппаратной платформы, на которой будет использоваться GNS3, возможно построение комплексных проектов, состоящих из маршрутизаторов Cisco, Cisco ASA, Juniper, а также серверов под управлением различных сетевых операционных систем.

В качестве альтернатив проекту GNS3 можно назвать другой современный симулятор -- IOU (IOS on UNIX). Этот эмулятор обладает практически полноценной поддержкой как L3, так и L2, а также используется при сдаче лабораторных экзаменов CCIE. Но это проприетарный софт, который официально не подлежит распространению вообще (на торрентах можно найти образы L2IOU и L3IOU, но это не законно). Кроме того, как видно из названия, работает он только на Linux.

В данной работе рассматривается пример построения сети с простейшей топологией, подключение двух виртуальных машин к этой сети и мониторинг сетевого трафика средствами WireShark.