\newpage
\section*{Введение}
\addcontentsline{toc}{section}{Введение}

Настоящая лабораторная работа расчитана на индивидуальное выполнение, с целью развития базовых навыков планирования и запуска сетей на основе TCP/IP протокола. Субьектом исследования будет являться полунатуральный эмулятор корпоративной компьютерной сети (ККС), который будет построен в виртуальном окружении.

Инструментальной платформой полунатурного эмулятора ККС служит программный продукт Oracle VM VirtualBox, разработанный компанией Oracle (ранее Sun Microsystems и Innotek). На момент написания отчёта (апрель 2015 года) актуальной являлась версия VirtualBox 4.3.26.

Ключевыми особенностями программы является:
\begin{itemize}
\item Кроссплатформенность;
\item Модульность;
\item Поддержка USB 2.0;
\item Поддержка 64-битных гостевых систем;
\item Поддержка SMP на стороне гостевой системы;
\item Встроенный RDP-сервер;
\item Поддержка аппаратного 3D-ускорения (OpenGL, DirectX);
\item Поддержка образов жёстких дисков VMDK, VHD и прочее;
\item Поддержка цепочки сохраненных состояний виртуальной машины (snapshots);
\item Поддержка iSCSI;
\item Поддержка виртуализации аудиоустройств;
\item Поддержка Shared Folders для простого обмена файлами между хостовой и гостевой системами;
\item Поддержка интеграции рабочих столов (seamless mode);
\item Мультиязычный интерфейс.
\end{itemize}

Отдельно стоит указать про сетевые возможности. VirtualBox обеспечивает следующие типы подключения:
\begin{itemize}
{\item Not attached (Сеть не подключена)

Режим в котором VirtualBox «сообщает» виртуальной машине, что сетевой интерфейс установлен, но кабель не подключен и сеть не доступна. Рекомендуется для отладки сетевых приложений.}

{\item Network Address Translation (NAT)

Наиболее простой способ получить доступ к внешней сети из виртуальной машины. Обычно не требует настроек, данный режим устанавливается для виртуальных сетевых интерфейсов по умолчанию. Виртуальная машина подключается к интернету, как реальная в локальной сети через маршрутизатор (шлюз, которым является для нее VirtualBox), но в данном режиме гостевая ОС не доступна для внешней сети.

Виртуальная машина получает IP адрес и настройки локальной сети через DHCP встроенный в VirtualBox. IP адрес хоста и гостя обычно находятся в разных подсетях. Первый сетевой интерфейс входит в сеть 10.0.2.0, второй в 10.0.3.0 и так далее. 

В данном режиме хосту не доступна внутренняя сеть и сетевые сервисы в виртуальной машине. Однако возможно используя перенаправление портов (port forwarding) сделать доступными сервисы виртуальной машины для хоста и других систем внешней сети - VirtualBox "слушает" порты на хосте и пересылает пакеты на виртуальный интерфейс. Для этого нужно учесть и распределить номера портов которые должен обрабатывать хост, а какие виртуальная машина (не всегда возможно использовать зарезервированные номера портов для нужного сервиса).


NAT имеет четыре существенных ограничения о которых нужно знать:
	\begin{itemize}
	\item Слабая поддержка ICMP, которая требуется многим программам мониторинга и отладки (например ping 	или tracerouting);
	\item Госевые ОС не прослушивают постоянно широковещательные рассылки, это сделано для улучшения производительности (как следствие, протокол NetBios не всегда корректно работает);
	\item Не поддерживаются некоторые протоколы, такие как GRE (это означает, что VPN (PPTP от Microsoft) не может использоваться);
	\item На Unix-based хостах (Linux, Solaris, MacOS X и т.п.) невозможно настроить порты ниже 1024 из приложений которые не запущены с правами суперпользователя (root). 
	\end{itemize}
}

{\item Bridged networking (Сетевой мост)

В этом режиме VirtualBox использует драйвер сетевого устройства в системе хоста для обработки пакетов с реального сетевого интерфейса. Этот драйвер называют «net filter» (сетевой фильтр). Он позволяет перехватывать пакеты из физической сети и создавать новые «программные» сетевые интерфейсы. При использовании виртуальной машиной такого программного интерфейса, возможна работа этого виртуального интерфейса через сетевой интерфейс хоста. Хост может принимать и посылать данные гостевой ОС, что означает что ВМ может взаимодействовать с другими устройствами физической сети.
}

{\item Internal networking (Внутренняя сеть)


Внутренняя сеть подобна обычной физической сети , в которой виртуальная машина может непосредственно общаться с внешним миром. Однако, "внешний мир" ограничен другими виртуальными машинами, которые соединяются к той же самой внутренней сети. Есть несколько оснований использовать данное решение:
	\begin{enumerate}
	\item Безопасность. В режиме сетевого моста весь сетевой трафик проходит через физический интерфейс системного хоста. Поэтому сетевые анализаторы (net sniffer) могут могут регистрировать сетевой трафик ВМ. Если необходимо, чтобы две или более ВМ на той же самом хосте общались конфиденциально, скрывая свои данные и от хоста и от пользователя, то это решение вполне подойдет.
	\item Скорость. Режим внутренней сети более эффективен, чем сетевой мост, поскольку VirtualBox может непосредственно передавать данные — нет необходимость посылать это данные через сетевой стек операционной системы хоста.
	\end{enumerate}

Внутренние сети создаются автоматически когда вам необходимо, то есть не нужды их настраивать. Каждая внутренняя сеть идентифицируется своим именем (задается произвольно).
}

{\item Host-only networking (виртуальный адаптер хоста)

Данный режим появился начиная с версией 2.2. Он проявляется как нечто среднее между режимами «сетевой мост» и «внутренняя сеть»: ВМ могут «общаться» друг с другом и хостом, но не могут взаимодействовать с внешней сетью хоста, так как они не связаны с его физическим сетевым интерфейсом.

VirtualBox создает новый «программный» интерфейс на хосте, который «выглядит» так же как и реально существующие сетевые интерфейсы хоста. Другими словами, в режиме «сетевой мост» используется существующий физический интерфейса, а в режиме «виртуальный адаптер хоста» на хоста создается интерфейс типа "петля" (loopback). И как с случае с «внутренняя сеть», трафик между ВМ не может быть перехвачен, но трафик на «петлевом» интерфейсе хоста перехватить можно. Данный режим полезно использовать для организации «изолированных» систем из нескольких ВМ настроенных на совместное функционирование. Например, одна ВМ может содержать web-сервер, а вторая СУБД, тогда web-сервер можно настроить на обработку запросов из физической сети(типа DMZ), а сервер СУБД будет изолирован от нее.
}
\end{itemize}

По умолчанию, устанавливается NAT, так как он удовлетворяет практически все стандартные сетевых потребности пользователя.