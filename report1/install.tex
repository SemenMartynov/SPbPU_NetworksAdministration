\newpage
\section{Инсталляция инструментальной среды}

Установка VirtualBox возможна из репозиториев ubuntu либо из репозиториев Oracle.

\textbf{Из репозиториев ubuntu}

Для установки необходимо в терминале набрать следующую команду:

\begin{Verbatim}[frame=single]
sudo apt-get install virtualbox
\end{Verbatim}

Для продолжения операции у Вас будет запрошен пароль, введите Ваш пароль и ждите пока закончится загрузка и установка приложения.

\textbf{Из репозиториев Oracle}

Версию VirtualBox можно установить с официального репозитория Oracle. На нём находятся более новые версии.

Для добавления репозитория нужно воспользоваться терминалом.

Необходимо добавить официальный репозиторий VirtualBox в файл /etc/apt/sources.list . Для этого выполните команду:

\begin{Verbatim}[frame=single]
echo "deb http://download.virtualbox.org/virtualbox/debian \
 $(lsb_release -sc) contrib" | sudo tee -a /etc/apt/sources.list
\end{Verbatim}

Добавим и зарегистрируем в системе ключ репозитория с помощью команды в терминал:

\begin{Verbatim}[frame=single]
wget -q https://www.virtualbox.org/download/oracle_vbox.asc -O- | \
                                                          sudo apt-key add -
\end{Verbatim}

Вы должны увидеть примерно следующий текст в Источниках приложений в „Аутентификации”:

\begin{Verbatim}[frame=single]
7B0F AB3A 13B9 0743 5925  D9C9 5442 2A4B 98AB 5139
Oracle Corporation (VirtualBox archive signing key) <info@virtualbox.org>
\end{Verbatim}

Обновите список пакетов:

\begin{Verbatim}[frame=single]
sudo apt-get update
\end{Verbatim}

Устанавливаем пакет для модулей ядра таких как vboxdrv и vboxnetflt:

\begin{Verbatim}[frame=single]
sudo apt-get install dkms
\end{Verbatim}

Для установки VirtualBox введите:

\begin{Verbatim}[frame=single]
sudo apt-get install virtualbox-4.3
\end{Verbatim}

Если нужна более старая версия: замените virtualbox-4.3 на:

\begin{Verbatim}[frame=single]
virtualbox-4.2 для установки VirtualBox 4.2.20
virtualbox-4.1 для установки VirtualBox 4.1.28
\end{Verbatim}

После того как VirtualBox установится, вам нужно добавить вашего пользователя в группу vboxusers. Для этого выполните команду в терминале:

\begin{Verbatim}[frame=single]
sudo usermod -a -G vboxusers `whoami`
\end{Verbatim}

Для применения изменений необходимо завершить сеанс и повторить вход в систему, либо перезагрузиться.