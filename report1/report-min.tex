\documentclass[a4paper, 12pt]{article}		% general format

%%%% Charset
\usepackage{cmap}							% make PDF files searchable and copyable
\usepackage[utf8x]{inputenc}				% accept different input encodings
\usepackage[T2A]{fontenc}					% russian font
\usepackage[russian]{babel}					% multilingual support (T2A)

%%%% Graphics
\usepackage[dvipsnames]{xcolor}			% driver-independent color extensions
\usepackage{graphicx}						% enhanced support for graphics
\usepackage{wrapfig}						% produces figures which text can flow around

%%%% Math
\usepackage{amsmath}						% American Mathematical Society (AMS) math facilities
\usepackage{amsfonts}						% fonts from the AMS
\usepackage{amssymb}						% additional math symbols

%%%% Typograpy (don't forget about cm-super)
\usepackage{microtype}						% subliminal refinements towards typographical perfection
\linespread{1.3}							% line spacing
\usepackage[left=2.5cm, right=1.5cm, top=2.5cm, bottom=2.5cm]{geometry}
\setlength{\parindent}{0pt}					% we don't want any paragraph indentation
\usepackage{parskip}						% some distance between paragraphs

%%%% Tables
\usepackage{tabularx}						% tables with variable width columns
\usepackage{multirow}						% for tabularx
\usepackage{hhline}							% for tabularx

%%%% Graph
\usepackage{tikz}							% package for creating graphics programmatically
\usetikzlibrary{arrows}						% edges for tikz

%%%% Other
\usepackage{url}							% verbatim with URL-sensitive line breaks
\usepackage{fancyvrb}						% sophisticated verbatim text (with box)
%------------------------------------------------------------------------------

\begin{document}

%------------------------------------------------
\begin{titlepage}
\thispagestyle{empty}

\begin{center}
Санкт-Петербургский политехнический университет Петра Великого\\
Институт Информационных Технологий и Управления \\*
Кафедра компьютерных систем и программных технологий \\*
\hrulefill
\end{center}

\vspace{15em}

\begin{center}
\Large Отчёт по лабораторной работе №1\\по предмету «Администрирование компьютерных сетей» \\
\end{center}

\vspace{1em}

% \linebreak
\begin{center}
\textsc{\textbf{Виртуальное макетирование компьютерных сетей}}
\end{center}

\vspace{20em}

\begin{flushleft}
Работу выполнил студент гр. 53501/3 \hrulefill Мартынов С. А. \\
\vspace{1.5em}
Работу принял преподаватель \hrulefill Малышев И. А. \\
\end{flushleft}

\vspace{\fill}

\begin{center}
Санкт-Петербург \\
2015
\end{center}

\end{titlepage}
%------------------------------------------------
\setcounter{page}{2}

%------------------------------------------------------------------------------

\textbf{Подготовка тестового стенда}

Тестовый стенд был построен на основе VirtualBox-4.3. Для установки этого продукта в терминале необходимо набрать следующую команду:

\begin{Verbatim}[frame=single]
sudo apt-get install virtualbox
\end{Verbatim}

\textbf{Разворачивание ККС}

Эмулируемая ККС имеет три сегмента (подсети):
\begin{itemize}
\item VMnet1 (адрес подсети – 192.168.40.0)
\item VMnet2 (адрес подсети – 192.168.80.0)
\item VMnet3 (адрес подсети – 192.168.120.0)
\end{itemize}

Сеть VMnet1 содержит основной многофункциональный сервер ККС (службы DHCP, DNS, Web, Proxy, почтовая, файловая и др.), использующий ОС Linux Mandrake. Сеть VMnet2 представлена хостом под управлением ОС Windows 2k (2000/XP/2003 Server) и имеет двойное назначение – она обеспечивает ККС дополнительным сервером и поддерживает Windows-клиент семейства 2k. Сеть VMnet3 объединяет рабочие станции c ограниченными аппаратными ресурсами, ориентированные на ОС Windows семейства 9x (95/98).

Взаимодействие сегментов ККС осуществляет маршрутизатор, обслуживаемый ОС семейства Unix (FreeBSD), а подключение ККС к внешней сети (Интернет или реальной сети учебной лаборатории) обеспечивает шлюз, реализуемый другой ОС семейства Unix (NetBSD). Каждый представитель подсетей (VMnet1, Vmnet2, VMnet3) имеет один сетевой адаптер, шлюз – два, а маршрутизатор – три сетевых адаптера.

Хост WIN98 (здесь и далее имя хоста формируется от названия гостевой ОС) имеет статический адрес 192.168.120.15, хост LINUX также имеет статический адрес – 192.168.40.32, а хост WIN2000 получает адрес 192.168.80.128 динамически с помощью виртуального сервера DHCP (встроенного по умолчанию в каждый сегмент сети).Более подробные данные о структуре ККС находятся в методическом пособии.

Эти данные задаются через графический интерфейс VirtualBox. Для организации DHCP в сети VMnet2, к сожалению, нет GUI-инструмента, но эта задача решается через командную строку:

\begin{Verbatim}[frame=single]
VBoxManage dhcpserver add --netname VMnet2 --ip 192.168.40.1 \
   --netmask 255.255.255.0 --lowerip 192.168.40.32 --enable
\end{Verbatim}

Удалить этот DHCP-сервер можно командой

\begin{Verbatim}[frame=single]
VBoxManage dhcpserver remove --netname VMnet2
\end{Verbatim}

\textbf{Настройка хостов}

В процессе установки, часть машин получило динамические адреса, часть статические (в соответствии со схемой).

Интерес представляет второйсетевой интерфейс NetBSDЮ его конфигурация определена в файле в файл /etc/rc.conf
\begin{Verbatim}[frame=single]
ifconfig_wm1="inet 192.168.40.57 netmask 255.255.255.0"
\end{Verbatim}

Для включения ip-форвардинга, в файл /etc/sysctl.conf нужно добавить строку
\begin{Verbatim}[frame=single]
net.inet.ip.forwarding=1
\end{Verbatim}

Потребовалось настроить некоторые демоны, такие как ipfilter (фаервол работает в режиме полного доступа, т.к. реальные задачи по ограничению доступа должны лежать на машине FreeBSD, выполняющей функции шлюза в локальную сеть) и ipnat (выполняет nat пакетов, предназначенных web-серверу и маскарадинг всех исходящих пакетов).

NetBSD является шлюзом по умолчанию для нескольких хостов, и чтобы система знала куда отвечать, на этой машине нужно добавить два маршрута (один для Win2000 и один для Win98):
\begin{Verbatim}[frame=single]
route add -net 192.168.80.0 -netmask 255.255.255.0 192.168.40.2
route add -net 192.168.120.0 -netmask 255.255.255.0 192.168.40.2
\end{Verbatim}

Кроме того, на NetBSD ещё нужно включить NAT для сети VMnet2 и VMnet3:
\begin{Verbatim}[frame=single]
echo "map wm0 192.168.80.0/24 -> 192.168.32.5/32" >> /etc/ipnat.conf
echo "map wm0 192.168.120.0/24 -> 192.168.32.5/32" >> /etc/ipnat.conf
\end{Verbatim}

\end{document}